\section{Code Sample}

\begin{lstlisting}[ style=cstyle,
                    caption={C Code}, 
                    label=lst:cExample ]
#include "functions.h"

// Constant matrices
const float L[3] = { -11.0, -12.0, -13.0 };
const float B1[4] = { 0.0, -0.2396, 0.0, 0.2396 };
const float B2[4] = { 0.2396, 0.0, -0.2396, 0.0 };
const float B3[4] = { 0.0377, -0.0377, 0.0377, -0.0377 };
\end{lstlisting}

In \autoref{lst:cExample} is some C-code, and here is some in-line C-code: \inlinec{xTaskCreate();}.

\begin{lstlisting}[ style=pythonstyle,
                    caption={Python Code}, 
                    label=lst:pythonExample ]
# This parses the packets to identify messages and decodes them for the logs
class packetParser():
    def __init__(self,accelfile,gpsfile,measstate,fulllog,plog):
        self.GPS = {0: 'Latitude',
                    1: 'Longtitude',
                    2: 'Velocity'}
        self.IMU = {0: 'AccelerationX',
                    1: 'AccelerationY',
                    2: 'AccelerationZ',
                    3: 'GyroscopeX',
                    4: 'GyroscopeY',
                    5: 'GyroscopeZ',
                    6: 'MagnetometerX',
                    7: 'MagnetometerY',
                    8: 'MagnetometerZ',
                    9: 'Temperature'}
        self.MsgID = {0: self.GPS, 1: self.IMU}
        self.DevID = {0: 'GPS', 1: 'IMU'}
        self.accelburst = [0,0,0,0,0,0,0]
        self.accellog = accelfile
        self.fulllog = fulllog
\end{lstlisting}

In \autoref{lst:pythonExample} is some Python-code, and here is some in-line Python-code:\\ \inlinepython{self.plog.write(str(msgnr))}

\begin{lstlisting}[ style=matlabstyle,
                    caption={Matlab Code}, 
                    label=lst:matlabExample ]
  close all
  clear
  clc
  
  % Parameters
  mx=200;     % [kg] mass + added mass in xb direction
  my=250;     % [kg] mass + added mass in yb direction
  Iz=700;     % [kgm2]
  
  dx=70;      % [kg/s] 
  dy=100;     % [kg/s]
  dyaw=50;    % [kgm2/s]
\end{lstlisting}

In \autoref{lst:cExample}, \ref{lst:pythonExample} and \ref{lst:matlabExample} is some code, and here is some in-line matlab: \inlinematlab{randn(50)}