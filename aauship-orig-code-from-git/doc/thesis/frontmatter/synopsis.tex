This master's thesis concerns the further work on the platform named
AAUSHIP. This is an \acf{ASV} which can have different
purposes.  Within the scope of this project it will be used for
surveying applications, where the AAUSHIP will be expanded to be a
fleet of AAUSHIPs to navigate in a formation.

Firstly, the old
AAUSHIP is upgraded with respect of hardware and implementation of these
have been necessary. The single AAUSHIP have been tested after newly
implemented Kalman Filter and a heading controller have been
implemented. This is used as a \acf{LOS} guidance to make the AAUSHIP
converge onto a predetermined trajectory. Afterwards is the focus to
investigate formation strategies to be implemented at the AAUSHIP when
more ships are to come. The main investigated strategy is based on a
potential field algorithm, which have been simulated with the dynamics
of the AAUSHIP. This is the basis of future work to implement this
strategy at the coming AAUSHIP fleet for verification of the methods.

Results show that, with the model designed, it is possible to control
the AAUSHIP in the area of interest. Further work to the model can
improve performance, but this has not been the main focus within the
scope of this project, where the surveying purpose have been in
focus. Simulation of the formation control strategy with potential
field shows the potential to implement this at the coming AAUSHIP
fleet, such that these will be able to perform surveying as an entire
group of vessels.
