Dette kandidatspeciale omhandler fortsat arbejde på AAUSHIP platformen. AAUSHIP er en \ac{ASV} som kan have flere forskellige anvendelsesmuligheder. Fokus i dette projekt er formationskontrol med baggrund i pejlingsopgaver hvortil AAUSHIP skal udvides fra et enkelt skib til en flåde af AAUSHIPs.

Båden er fysisk blevet opgraderet med nyt elektronik som er implementeret. Dertil er den enkelte båd blevet testet med et nyt \ac{KF} hvortil en retningsregulator også er implementeret. Denne er anvendt som en \ac{LOS} reference regulator for at få båden til at konvergere til de genererede liniestykker mellem rutepunkter. Efterfølgende har fokus været sat på at identificere og analysere formationsstrategier som skal implementeres på bådene, når disse er produceret. Mest fokus er lagt på en potentialefeltsalgoritme, som er blevet simuleret inklusiv dynamikken fra modellen af AAUSHIP. Dette danner grundlaget for implementering på flåden med opfølgende verifikation.

Resultater viser, at det er muligt at kontrollere AAUSHIP med den designede model, i et område givet af Aalborg Havn. Yderligere arbejde vil ligge i at forbedre modellen af AAUSHIP, men dette har ikke vist sig at være nødvendigt da fokus i dette project har omhandlet pejlingsopgaven. Slutteligt udviser simulering af formationskontrolstrategien potentiale for en mulig implementering af denne på den kommende flåde af AAUSHIPs, så disse kan foretage pejligsopgaver som en dannet formation af skibe.